% Generated by Sphinx.
\def\sphinxdocclass{report}
\documentclass[letterpaper,10pt,english]{sphinxmanual}
\usepackage[utf8]{inputenc}
\DeclareUnicodeCharacter{00A0}{\nobreakspace}
\usepackage{cmap}
\usepackage[T1]{fontenc}
\usepackage{babel}
\usepackage{times}
\usepackage[Bjarne]{fncychap}
\usepackage{longtable}
\usepackage{sphinx}
\usepackage{multirow}


\title{proto\_err Documentation}
\date{December 09, 2013}
\release{1}
\author{Phelim Bradley}
\newcommand{\sphinxlogo}{}
\renewcommand{\releasename}{Release}
\makeindex

\makeatletter
\def\PYG@reset{\let\PYG@it=\relax \let\PYG@bf=\relax%
    \let\PYG@ul=\relax \let\PYG@tc=\relax%
    \let\PYG@bc=\relax \let\PYG@ff=\relax}
\def\PYG@tok#1{\csname PYG@tok@#1\endcsname}
\def\PYG@toks#1+{\ifx\relax#1\empty\else%
    \PYG@tok{#1}\expandafter\PYG@toks\fi}
\def\PYG@do#1{\PYG@bc{\PYG@tc{\PYG@ul{%
    \PYG@it{\PYG@bf{\PYG@ff{#1}}}}}}}
\def\PYG#1#2{\PYG@reset\PYG@toks#1+\relax+\PYG@do{#2}}

\expandafter\def\csname PYG@tok@gd\endcsname{\def\PYG@tc##1{\textcolor[rgb]{0.63,0.00,0.00}{##1}}}
\expandafter\def\csname PYG@tok@gu\endcsname{\let\PYG@bf=\textbf\def\PYG@tc##1{\textcolor[rgb]{0.50,0.00,0.50}{##1}}}
\expandafter\def\csname PYG@tok@gt\endcsname{\def\PYG@tc##1{\textcolor[rgb]{0.00,0.27,0.87}{##1}}}
\expandafter\def\csname PYG@tok@gs\endcsname{\let\PYG@bf=\textbf}
\expandafter\def\csname PYG@tok@gr\endcsname{\def\PYG@tc##1{\textcolor[rgb]{1.00,0.00,0.00}{##1}}}
\expandafter\def\csname PYG@tok@cm\endcsname{\let\PYG@it=\textit\def\PYG@tc##1{\textcolor[rgb]{0.25,0.50,0.56}{##1}}}
\expandafter\def\csname PYG@tok@vg\endcsname{\def\PYG@tc##1{\textcolor[rgb]{0.73,0.38,0.84}{##1}}}
\expandafter\def\csname PYG@tok@m\endcsname{\def\PYG@tc##1{\textcolor[rgb]{0.13,0.50,0.31}{##1}}}
\expandafter\def\csname PYG@tok@mh\endcsname{\def\PYG@tc##1{\textcolor[rgb]{0.13,0.50,0.31}{##1}}}
\expandafter\def\csname PYG@tok@cs\endcsname{\def\PYG@tc##1{\textcolor[rgb]{0.25,0.50,0.56}{##1}}\def\PYG@bc##1{\setlength{\fboxsep}{0pt}\colorbox[rgb]{1.00,0.94,0.94}{\strut ##1}}}
\expandafter\def\csname PYG@tok@ge\endcsname{\let\PYG@it=\textit}
\expandafter\def\csname PYG@tok@vc\endcsname{\def\PYG@tc##1{\textcolor[rgb]{0.73,0.38,0.84}{##1}}}
\expandafter\def\csname PYG@tok@il\endcsname{\def\PYG@tc##1{\textcolor[rgb]{0.13,0.50,0.31}{##1}}}
\expandafter\def\csname PYG@tok@go\endcsname{\def\PYG@tc##1{\textcolor[rgb]{0.20,0.20,0.20}{##1}}}
\expandafter\def\csname PYG@tok@cp\endcsname{\def\PYG@tc##1{\textcolor[rgb]{0.00,0.44,0.13}{##1}}}
\expandafter\def\csname PYG@tok@gi\endcsname{\def\PYG@tc##1{\textcolor[rgb]{0.00,0.63,0.00}{##1}}}
\expandafter\def\csname PYG@tok@gh\endcsname{\let\PYG@bf=\textbf\def\PYG@tc##1{\textcolor[rgb]{0.00,0.00,0.50}{##1}}}
\expandafter\def\csname PYG@tok@ni\endcsname{\let\PYG@bf=\textbf\def\PYG@tc##1{\textcolor[rgb]{0.84,0.33,0.22}{##1}}}
\expandafter\def\csname PYG@tok@nl\endcsname{\let\PYG@bf=\textbf\def\PYG@tc##1{\textcolor[rgb]{0.00,0.13,0.44}{##1}}}
\expandafter\def\csname PYG@tok@nn\endcsname{\let\PYG@bf=\textbf\def\PYG@tc##1{\textcolor[rgb]{0.05,0.52,0.71}{##1}}}
\expandafter\def\csname PYG@tok@no\endcsname{\def\PYG@tc##1{\textcolor[rgb]{0.38,0.68,0.84}{##1}}}
\expandafter\def\csname PYG@tok@na\endcsname{\def\PYG@tc##1{\textcolor[rgb]{0.25,0.44,0.63}{##1}}}
\expandafter\def\csname PYG@tok@nb\endcsname{\def\PYG@tc##1{\textcolor[rgb]{0.00,0.44,0.13}{##1}}}
\expandafter\def\csname PYG@tok@nc\endcsname{\let\PYG@bf=\textbf\def\PYG@tc##1{\textcolor[rgb]{0.05,0.52,0.71}{##1}}}
\expandafter\def\csname PYG@tok@nd\endcsname{\let\PYG@bf=\textbf\def\PYG@tc##1{\textcolor[rgb]{0.33,0.33,0.33}{##1}}}
\expandafter\def\csname PYG@tok@ne\endcsname{\def\PYG@tc##1{\textcolor[rgb]{0.00,0.44,0.13}{##1}}}
\expandafter\def\csname PYG@tok@nf\endcsname{\def\PYG@tc##1{\textcolor[rgb]{0.02,0.16,0.49}{##1}}}
\expandafter\def\csname PYG@tok@si\endcsname{\let\PYG@it=\textit\def\PYG@tc##1{\textcolor[rgb]{0.44,0.63,0.82}{##1}}}
\expandafter\def\csname PYG@tok@s2\endcsname{\def\PYG@tc##1{\textcolor[rgb]{0.25,0.44,0.63}{##1}}}
\expandafter\def\csname PYG@tok@vi\endcsname{\def\PYG@tc##1{\textcolor[rgb]{0.73,0.38,0.84}{##1}}}
\expandafter\def\csname PYG@tok@nt\endcsname{\let\PYG@bf=\textbf\def\PYG@tc##1{\textcolor[rgb]{0.02,0.16,0.45}{##1}}}
\expandafter\def\csname PYG@tok@nv\endcsname{\def\PYG@tc##1{\textcolor[rgb]{0.73,0.38,0.84}{##1}}}
\expandafter\def\csname PYG@tok@s1\endcsname{\def\PYG@tc##1{\textcolor[rgb]{0.25,0.44,0.63}{##1}}}
\expandafter\def\csname PYG@tok@gp\endcsname{\let\PYG@bf=\textbf\def\PYG@tc##1{\textcolor[rgb]{0.78,0.36,0.04}{##1}}}
\expandafter\def\csname PYG@tok@sh\endcsname{\def\PYG@tc##1{\textcolor[rgb]{0.25,0.44,0.63}{##1}}}
\expandafter\def\csname PYG@tok@ow\endcsname{\let\PYG@bf=\textbf\def\PYG@tc##1{\textcolor[rgb]{0.00,0.44,0.13}{##1}}}
\expandafter\def\csname PYG@tok@sx\endcsname{\def\PYG@tc##1{\textcolor[rgb]{0.78,0.36,0.04}{##1}}}
\expandafter\def\csname PYG@tok@bp\endcsname{\def\PYG@tc##1{\textcolor[rgb]{0.00,0.44,0.13}{##1}}}
\expandafter\def\csname PYG@tok@c1\endcsname{\let\PYG@it=\textit\def\PYG@tc##1{\textcolor[rgb]{0.25,0.50,0.56}{##1}}}
\expandafter\def\csname PYG@tok@kc\endcsname{\let\PYG@bf=\textbf\def\PYG@tc##1{\textcolor[rgb]{0.00,0.44,0.13}{##1}}}
\expandafter\def\csname PYG@tok@c\endcsname{\let\PYG@it=\textit\def\PYG@tc##1{\textcolor[rgb]{0.25,0.50,0.56}{##1}}}
\expandafter\def\csname PYG@tok@mf\endcsname{\def\PYG@tc##1{\textcolor[rgb]{0.13,0.50,0.31}{##1}}}
\expandafter\def\csname PYG@tok@err\endcsname{\def\PYG@bc##1{\setlength{\fboxsep}{0pt}\fcolorbox[rgb]{1.00,0.00,0.00}{1,1,1}{\strut ##1}}}
\expandafter\def\csname PYG@tok@kd\endcsname{\let\PYG@bf=\textbf\def\PYG@tc##1{\textcolor[rgb]{0.00,0.44,0.13}{##1}}}
\expandafter\def\csname PYG@tok@ss\endcsname{\def\PYG@tc##1{\textcolor[rgb]{0.32,0.47,0.09}{##1}}}
\expandafter\def\csname PYG@tok@sr\endcsname{\def\PYG@tc##1{\textcolor[rgb]{0.14,0.33,0.53}{##1}}}
\expandafter\def\csname PYG@tok@mo\endcsname{\def\PYG@tc##1{\textcolor[rgb]{0.13,0.50,0.31}{##1}}}
\expandafter\def\csname PYG@tok@mi\endcsname{\def\PYG@tc##1{\textcolor[rgb]{0.13,0.50,0.31}{##1}}}
\expandafter\def\csname PYG@tok@kn\endcsname{\let\PYG@bf=\textbf\def\PYG@tc##1{\textcolor[rgb]{0.00,0.44,0.13}{##1}}}
\expandafter\def\csname PYG@tok@o\endcsname{\def\PYG@tc##1{\textcolor[rgb]{0.40,0.40,0.40}{##1}}}
\expandafter\def\csname PYG@tok@kr\endcsname{\let\PYG@bf=\textbf\def\PYG@tc##1{\textcolor[rgb]{0.00,0.44,0.13}{##1}}}
\expandafter\def\csname PYG@tok@s\endcsname{\def\PYG@tc##1{\textcolor[rgb]{0.25,0.44,0.63}{##1}}}
\expandafter\def\csname PYG@tok@kp\endcsname{\def\PYG@tc##1{\textcolor[rgb]{0.00,0.44,0.13}{##1}}}
\expandafter\def\csname PYG@tok@w\endcsname{\def\PYG@tc##1{\textcolor[rgb]{0.73,0.73,0.73}{##1}}}
\expandafter\def\csname PYG@tok@kt\endcsname{\def\PYG@tc##1{\textcolor[rgb]{0.56,0.13,0.00}{##1}}}
\expandafter\def\csname PYG@tok@sc\endcsname{\def\PYG@tc##1{\textcolor[rgb]{0.25,0.44,0.63}{##1}}}
\expandafter\def\csname PYG@tok@sb\endcsname{\def\PYG@tc##1{\textcolor[rgb]{0.25,0.44,0.63}{##1}}}
\expandafter\def\csname PYG@tok@k\endcsname{\let\PYG@bf=\textbf\def\PYG@tc##1{\textcolor[rgb]{0.00,0.44,0.13}{##1}}}
\expandafter\def\csname PYG@tok@se\endcsname{\let\PYG@bf=\textbf\def\PYG@tc##1{\textcolor[rgb]{0.25,0.44,0.63}{##1}}}
\expandafter\def\csname PYG@tok@sd\endcsname{\let\PYG@it=\textit\def\PYG@tc##1{\textcolor[rgb]{0.25,0.44,0.63}{##1}}}

\def\PYGZbs{\char`\\}
\def\PYGZus{\char`\_}
\def\PYGZob{\char`\{}
\def\PYGZcb{\char`\}}
\def\PYGZca{\char`\^}
\def\PYGZam{\char`\&}
\def\PYGZlt{\char`\<}
\def\PYGZgt{\char`\>}
\def\PYGZsh{\char`\#}
\def\PYGZpc{\char`\%}
\def\PYGZdl{\char`\$}
\def\PYGZhy{\char`\-}
\def\PYGZsq{\char`\'}
\def\PYGZdq{\char`\"}
\def\PYGZti{\char`\~}
% for compatibility with earlier versions
\def\PYGZat{@}
\def\PYGZlb{[}
\def\PYGZrb{]}
\makeatother

\begin{document}

\maketitle
\tableofcontents
\phantomsection\label{index::doc}



\chapter{\texttt{proto\_err} -- Main package}
\label{index:proto-err-main-package}\label{index:welcome-to-proto-err-s-documentation}\phantomsection\label{index:module-proto_err}\index{proto\_err (module)}

\section{\texttt{proto\_err.simulation} -- Simulating Error Reads}
\label{index:proto-err-simulation-simulating-error-reads}\label{index:module-proto_err.simulation}\index{proto\_err.simulation (module)}\index{complexError (class in proto\_err.simulation)}

\begin{fulllineitems}
\phantomsection\label{index:proto_err.simulation.complexError}\pysiglinewithargsret{\strong{class }\code{proto\_err.simulation.}\bfcode{complexError}}{\emph{record}, \emph{opt}, \emph{id}}{}
Bases: {\hyperref[index:proto_err.simulation.simulateError]{\code{proto\_err.simulation.simulateError}}}
\paragraph{Methods}
\index{error() (proto\_err.simulation.complexError method)}

\begin{fulllineitems}
\phantomsection\label{index:proto_err.simulation.complexError.error}\pysiglinewithargsret{\bfcode{error}}{}{}
Function to induce and error

\end{fulllineitems}


\end{fulllineitems}

\index{simulateError (class in proto\_err.simulation)}

\begin{fulllineitems}
\phantomsection\label{index:proto_err.simulation.simulateError}\pysiglinewithargsret{\strong{class }\code{proto\_err.simulation.}\bfcode{simulateError}}{\emph{record}, \emph{opt}, \emph{id}}{}
Class of objects to simulate errors in a SeqRecord
\paragraph{Methods}
\index{deletion() (proto\_err.simulation.simulateError method)}

\begin{fulllineitems}
\phantomsection\label{index:proto_err.simulation.simulateError.deletion}\pysiglinewithargsret{\bfcode{deletion}}{\emph{pos}, \emph{dlen}}{}
function to induce a deletion

\end{fulllineitems}

\index{indel() (proto\_err.simulation.simulateError method)}

\begin{fulllineitems}
\phantomsection\label{index:proto_err.simulation.simulateError.indel}\pysiglinewithargsret{\bfcode{indel}}{\emph{pos}}{}
function to induce an INDEL

\end{fulllineitems}

\index{ins() (proto\_err.simulation.simulateError method)}

\begin{fulllineitems}
\phantomsection\label{index:proto_err.simulation.simulateError.ins}\pysiglinewithargsret{\bfcode{ins}}{\emph{pos}, \emph{rl}}{}
function to induce a insertion

\end{fulllineitems}

\index{qscore (proto\_err.simulation.simulateError attribute)}

\begin{fulllineitems}
\phantomsection\label{index:proto_err.simulation.simulateError.qscore}\pysigline{\bfcode{qscore}\strong{ None}}
\end{fulllineitems}

\index{record (proto\_err.simulation.simulateError attribute)}

\begin{fulllineitems}
\phantomsection\label{index:proto_err.simulation.simulateError.record}\pysigline{\bfcode{record}\strong{ None}}
Return a Bio.SeqRecord

\end{fulllineitems}

\index{snp() (proto\_err.simulation.simulateError method)}

\begin{fulllineitems}
\phantomsection\label{index:proto_err.simulation.simulateError.snp}\pysiglinewithargsret{\bfcode{snp}}{\emph{pos}, \emph{rl}}{}
function to induce a SNP

\end{fulllineitems}


\end{fulllineitems}

\index{singleSNP (class in proto\_err.simulation)}

\begin{fulllineitems}
\phantomsection\label{index:proto_err.simulation.singleSNP}\pysiglinewithargsret{\strong{class }\code{proto\_err.simulation.}\bfcode{singleSNP}}{\emph{record}, \emph{opt}, \emph{id}}{}
Bases: {\hyperref[index:proto_err.simulation.simulateError]{\code{proto\_err.simulation.simulateError}}}
\paragraph{Methods}
\index{error() (proto\_err.simulation.singleSNP method)}

\begin{fulllineitems}
\phantomsection\label{index:proto_err.simulation.singleSNP.error}\pysiglinewithargsret{\bfcode{error}}{}{}
Function to induce and error

\end{fulllineitems}


\end{fulllineitems}

\index{subsample() (in module proto\_err.simulation)}

\begin{fulllineitems}
\phantomsection\label{index:proto_err.simulation.subsample}\pysiglinewithargsret{\code{proto\_err.simulation.}\bfcode{subsample}}{\emph{ref}, \emph{opt}, \emph{errorSimulator=\textless{}class proto\_err.simulation.complexError at 0x10544c1f0\textgreater{}}}{}
Function to take a fasta file subsample reads and generate a list of 
subsampled reads

\end{fulllineitems}



\section{\texttt{proto\_err.errorCount} -- Error Counting}
\label{index:proto-err-errorcount-error-counting}\label{index:module-proto_err.errorCount}\index{proto\_err.errorCount (module)}\index{counter (class in proto\_err.errorCount)}

\begin{fulllineitems}
\phantomsection\label{index:proto_err.errorCount.counter}\pysiglinewithargsret{\strong{class }\code{proto\_err.errorCount.}\bfcode{counter}}{\emph{ref}, \emph{errorList=None}, \emph{samfile=None}, \emph{opt=None}}{}
Takes a list of errors and does some kmer counting
\paragraph{Methods}
\index{countAlignedBases() (proto\_err.errorCount.counter method)}

\begin{fulllineitems}
\phantomsection\label{index:proto_err.errorCount.counter.countAlignedBases}\pysiglinewithargsret{\bfcode{countAlignedBases}}{\emph{read}}{}
\end{fulllineitems}

\index{countErrorKmer() (proto\_err.errorCount.counter method)}

\begin{fulllineitems}
\phantomsection\label{index:proto_err.errorCount.counter.countErrorKmer}\pysiglinewithargsret{\bfcode{countErrorKmer}}{\emph{maxKmerLength=None}}{}
Count all kmers in the list of errors of length kmerLen or below before 
and after and error.
\begin{quote}\begin{description}
\item[{Parameters}] \leavevmode
\textbf{maxKmerLength} : int
\begin{quote}

This is a type.
Maximum kmer length
\end{quote}

\item[{Returns}] \leavevmode
\textbf{emmissionDic,beforeAfterCount} :
\begin{quote}

dict,dict
emmissionDic counts the number of times a transition occurs

e.g. \{`A':\{`T':123,'C':123,...,\},'T':\{`A':123,...\},...\}

emmissionDic{[}trueBase{]}{[}emmitedBase{]} = count of trueBase -\textgreater{} emmitedBase transition

beforeAfterCount counts the number of times a kmer appears before or after an error

beforeAfterCount{[}'before'{]}{[}trueBase{]}{[}emmitedBase{]}{[}kmer{]} = count of kmer occurance before trueBase -\textgreater{} emmitedBase transition

beforeAfterCount{[}'after'{]}{[}trueBase{]}{[}emmitedBase{]}{[}kmer{]} = count of kmer occurance after trueBase -\textgreater{} emmitedBase transition

These dictonaries are also stored in the counter object in counter.res{[}'errorMode'{]},counter.res{[}'kmerCounts'{]}
\end{quote}

\end{description}\end{quote}


\strong{See also:}


{\hyperref[index:proto_err.errorCount.counter.countRefKmer]{\code{countRefKmer}}}


\paragraph{Examples}

\begin{Verbatim}[commandchars=\\\{\}]
\PYG{g+gp}{\PYGZgt{}\PYGZgt{}\PYGZgt{} }\PYG{k+kn}{from} \PYG{n+nn}{errorCount} \PYG{k+kn}{import} \PYG{n}{counter}
\PYG{g+gp}{\PYGZgt{}\PYGZgt{}\PYGZgt{} }\PYG{n}{errorCounter} \PYG{o}{=} \PYG{n}{counter}\PYG{p}{(}\PYG{n}{ref}\PYG{p}{,}\PYG{n}{samfile}\PYG{p}{)}
\PYG{g+gp}{\PYGZgt{}\PYGZgt{}\PYGZgt{} }\PYG{n}{errorCounter}\PYG{o}{.}\PYG{n}{countErrorKmer}\PYG{p}{(}\PYG{n}{maxKmerLength} \PYG{o}{=} \PYG{l+m+mi}{3}\PYG{p}{)} 
\PYG{g+gp}{\PYGZgt{}\PYGZgt{}\PYGZgt{} }\PYG{c}{\PYGZsh{}\PYGZsh{} counts all possible kmers of length 1,2,3 before and after an error. }
\PYG{g+gp}{\PYGZgt{}\PYGZgt{}\PYGZgt{} }\PYG{p}{(}\PYG{p}{\PYGZob{}}\PYG{l+s}{\PYGZsq{}}\PYG{l+s}{A}\PYG{l+s}{\PYGZsq{}}\PYG{p}{:} \PYG{p}{\PYGZob{}}\PYG{l+s}{\PYGZsq{}}\PYG{l+s}{C}\PYG{l+s}{\PYGZsq{}}\PYG{p}{:} \PYG{l+m+mi}{113}\PYG{p}{,} \PYG{l+s}{\PYGZsq{}}\PYG{l+s}{T}\PYG{l+s}{\PYGZsq{}}\PYG{p}{:} \PYG{l+m+mi}{105}\PYG{p}{,} \PYG{l+s}{\PYGZsq{}}\PYG{l+s}{G}\PYG{l+s}{\PYGZsq{}}\PYG{p}{:} \PYG{l+m+mi}{84}\PYG{p}{\PYGZcb{}}\PYG{p}{,} \PYG{o}{.}\PYG{o}{.}\PYG{o}{.}\PYG{p}{\PYGZcb{}}\PYG{p}{,} \PYG{p}{\PYGZob{}}\PYG{l+s}{\PYGZsq{}}\PYG{l+s}{after}\PYG{l+s}{\PYGZsq{}}\PYG{p}{:}\PYG{p}{\PYGZob{}}\PYG{l+s}{\PYGZsq{}}\PYG{l+s}{A}\PYG{l+s}{\PYGZsq{}}\PYG{p}{:} \PYG{p}{\PYGZob{}}\PYG{l+s}{\PYGZsq{}}\PYG{l+s}{C}\PYG{l+s}{\PYGZsq{}}\PYG{p}{:} \PYG{p}{\PYGZob{}}\PYG{l+s}{\PYGZsq{}}\PYG{l+s}{CTT}\PYG{l+s}{\PYGZsq{}}\PYG{p}{:} \PYG{l+m+mi}{2}\PYG{p}{,} \PYG{l+s}{\PYGZsq{}}\PYG{l+s}{GCA}\PYG{l+s}{\PYGZsq{}}\PYG{p}{:} \PYG{l+m+mi}{1}\PYG{p}{,}\PYG{o}{.}\PYG{o}{.}\PYG{o}{.}\PYG{p}{\PYGZcb{}}\PYG{p}{,}\PYG{o}{.}\PYG{o}{.}\PYG{p}{\PYGZcb{}}\PYG{p}{\PYGZcb{}} \PYG{p}{,}\PYG{l+s}{\PYGZsq{}}\PYG{l+s}{before:\PYGZob{}}\PYG{l+s}{\PYGZsq{}}\PYG{n}{A}\PYG{l+s}{\PYGZsq{}}\PYG{l+s}{: \PYGZob{}}\PYG{l+s}{\PYGZsq{}}\PYG{n}{C}\PYG{l+s}{\PYGZsq{}}\PYG{l+s}{: \PYGZob{}}\PYG{l+s}{\PYGZsq{}}\PYG{n}{CTT}\PYG{l+s}{\PYGZsq{}}\PYG{l+s}{: 2, }\PYG{l+s}{\PYGZsq{}}\PYG{n}{GCA}\PYG{l+s}{\PYGZsq{}}\PYG{l+s}{: 1,...\PYGZcb{},..\PYGZcb{}\PYGZcb{}}\PYG{l+s}{\PYGZsq{}}\PYG{p}{\PYGZcb{}}\PYG{p}{)}
\end{Verbatim}

\end{fulllineitems}

\index{countRefKmer() (proto\_err.errorCount.counter method)}

\begin{fulllineitems}
\phantomsection\label{index:proto_err.errorCount.counter.countRefKmer}\pysiglinewithargsret{\bfcode{countRefKmer}}{\emph{maxKmerLength=None}}{}
Count all kmers in the reference of length kmerLen or below

i.e.  counter.countRefKmer(maxKmerLength = 3) counts all possible kmers of length 1,2,3

returns a dictonary of counts in the form

\{`AAT':123,'AA':123,...\}

This dictonary is also stored in the counter object in counter.res{[}'RefCounts'{]}

\end{fulllineitems}

\index{getCount() (proto\_err.errorCount.counter method)}

\begin{fulllineitems}
\phantomsection\label{index:proto_err.errorCount.counter.getCount}\pysiglinewithargsret{\bfcode{getCount}}{\emph{truth=None}, \emph{emission=None}, \emph{kmer='`}, \emph{after=False}}{}
Gets the count for a given \{truth,emmision,kmer\}

\end{fulllineitems}

\index{kmerFreq() (proto\_err.errorCount.counter method)}

\begin{fulllineitems}
\phantomsection\label{index:proto_err.errorCount.counter.kmerFreq}\pysiglinewithargsret{\bfcode{kmerFreq}}{\emph{seq}, \emph{kmer}}{}
Calculate the number of times a kmer appears in a sequence
seq : SeqRecord Object
kmer : kmer string

\end{fulllineitems}

\index{readDiff() (proto\_err.errorCount.counter method)}

\begin{fulllineitems}
\phantomsection\label{index:proto_err.errorCount.counter.readDiff}\pysiglinewithargsret{\bfcode{readDiff}}{\emph{read}, \emph{ref}}{}
\end{fulllineitems}

\index{setup() (proto\_err.errorCount.counter method)}

\begin{fulllineitems}
\phantomsection\label{index:proto_err.errorCount.counter.setup}\pysiglinewithargsret{\bfcode{setup}}{\emph{opt}}{}
Setup output

\end{fulllineitems}


\end{fulllineitems}

\index{error (class in proto\_err.errorCount)}

\begin{fulllineitems}
\phantomsection\label{index:proto_err.errorCount.error}\pysiglinewithargsret{\strong{class }\code{proto\_err.errorCount.}\bfcode{error}}{\emph{true}, \emph{emission}, \emph{read}, \emph{readPos}}{}
Information about the errors in a read
\paragraph{Methods}
\index{after() (proto\_err.errorCount.error method)}

\begin{fulllineitems}
\phantomsection\label{index:proto_err.errorCount.error.after}\pysiglinewithargsret{\bfcode{after}}{\emph{j}}{}
Return the following j bases,return N when bases missing
e.g. error.after(2) returns the 2 bases after the error

\end{fulllineitems}

\index{before() (proto\_err.errorCount.error method)}

\begin{fulllineitems}
\phantomsection\label{index:proto_err.errorCount.error.before}\pysiglinewithargsret{\bfcode{before}}{\emph{j}}{}
Return the preceding j bases,return N when bases missing
e.g. error.before(2) returns the 2 bases before the error
`'

\end{fulllineitems}

\index{errorType (proto\_err.errorCount.error attribute)}

\begin{fulllineitems}
\phantomsection\label{index:proto_err.errorCount.error.errorType}\pysigline{\bfcode{errorType}\strong{ None}}
The type of read error 
SNP, Insertion or Deletion

\end{fulllineitems}

\index{isDeletion (proto\_err.errorCount.error attribute)}

\begin{fulllineitems}
\phantomsection\label{index:proto_err.errorCount.error.isDeletion}\pysigline{\bfcode{isDeletion}\strong{ None}}
Is the error a deletion

\end{fulllineitems}

\index{isIndel (proto\_err.errorCount.error attribute)}

\begin{fulllineitems}
\phantomsection\label{index:proto_err.errorCount.error.isIndel}\pysigline{\bfcode{isIndel}\strong{ None}}
Is the error an INDEL

\end{fulllineitems}

\index{isInsertion (proto\_err.errorCount.error attribute)}

\begin{fulllineitems}
\phantomsection\label{index:proto_err.errorCount.error.isInsertion}\pysigline{\bfcode{isInsertion}\strong{ None}}
Is the error an insertion

\end{fulllineitems}

\index{isSnp (proto\_err.errorCount.error attribute)}

\begin{fulllineitems}
\phantomsection\label{index:proto_err.errorCount.error.isSnp}\pysigline{\bfcode{isSnp}\strong{ None}}
Is the error a SNP

\end{fulllineitems}

\index{qscore() (proto\_err.errorCount.error method)}

\begin{fulllineitems}
\phantomsection\label{index:proto_err.errorCount.error.qscore}\pysiglinewithargsret{\bfcode{qscore}}{\emph{i}}{}
Return the quality score at a base +i i from the error start position
error.qscore(0) is equivalent to error.qual

\end{fulllineitems}

\index{qual (proto\_err.errorCount.error attribute)}

\begin{fulllineitems}
\phantomsection\label{index:proto_err.errorCount.error.qual}\pysigline{\bfcode{qual}\strong{ None}}
Quality score of the base where the error occured. 
equivalent to error.qscore(0)

\end{fulllineitems}


\end{fulllineitems}

\index{errorReader (class in proto\_err.errorCount)}

\begin{fulllineitems}
\phantomsection\label{index:proto_err.errorCount.errorReader}\pysiglinewithargsret{\strong{class }\code{proto\_err.errorCount.}\bfcode{errorReader}}{\emph{samfile}, \emph{ref}}{}
Iterable over errors in aligned reads
\paragraph{Methods}
\index{checkDeletion() (proto\_err.errorCount.errorReader method)}

\begin{fulllineitems}
\phantomsection\label{index:proto_err.errorCount.errorReader.checkDeletion}\pysiglinewithargsret{\bfcode{checkDeletion}}{\emph{N}}{}
Checks Deletionread segment for  errors. called when cigarstring = D:N

\end{fulllineitems}

\index{checkHardClipped() (proto\_err.errorCount.errorReader method)}

\begin{fulllineitems}
\phantomsection\label{index:proto_err.errorCount.errorReader.checkHardClipped}\pysiglinewithargsret{\bfcode{checkHardClipped}}{\emph{N}}{}
Checks HardClipped read segment for errors. called when cigarstring = H:N

\end{fulllineitems}

\index{checkInsertion() (proto\_err.errorCount.errorReader method)}

\begin{fulllineitems}
\phantomsection\label{index:proto_err.errorCount.errorReader.checkInsertion}\pysiglinewithargsret{\bfcode{checkInsertion}}{\emph{N}}{}
Checks Insertion read segment for errors. called when cigarstring = I:N

\end{fulllineitems}

\index{checkPadding() (proto\_err.errorCount.errorReader method)}

\begin{fulllineitems}
\phantomsection\label{index:proto_err.errorCount.errorReader.checkPadding}\pysiglinewithargsret{\bfcode{checkPadding}}{\emph{N}}{}
Checks Padding read segment for errors. called when cigarstring = P:N

\end{fulllineitems}

\index{checkRead() (proto\_err.errorCount.errorReader method)}

\begin{fulllineitems}
\phantomsection\label{index:proto_err.errorCount.errorReader.checkRead}\pysiglinewithargsret{\bfcode{checkRead}}{}{}
Checks current read for errors

\end{fulllineitems}

\index{checkSNPs() (proto\_err.errorCount.errorReader method)}

\begin{fulllineitems}
\phantomsection\label{index:proto_err.errorCount.errorReader.checkSNPs}\pysiglinewithargsret{\bfcode{checkSNPs}}{\emph{N}}{}
Checks read segment for SNP errors. called when cigarstring = M:N

\end{fulllineitems}

\index{checkSeqMismatch() (proto\_err.errorCount.errorReader method)}

\begin{fulllineitems}
\phantomsection\label{index:proto_err.errorCount.errorReader.checkSeqMismatch}\pysiglinewithargsret{\bfcode{checkSeqMismatch}}{\emph{N}}{}
Checks SeqMismatch read segment for errors. called when cigarstring = =:N

\end{fulllineitems}

\index{checkSkipped() (proto\_err.errorCount.errorReader method)}

\begin{fulllineitems}
\phantomsection\label{index:proto_err.errorCount.errorReader.checkSkipped}\pysiglinewithargsret{\bfcode{checkSkipped}}{\emph{N}}{}
Checks Skipped read segment for errors. called when cigarstring = X:N

\end{fulllineitems}

\index{checkSoftClipped() (proto\_err.errorCount.errorReader method)}

\begin{fulllineitems}
\phantomsection\label{index:proto_err.errorCount.errorReader.checkSoftClipped}\pysiglinewithargsret{\bfcode{checkSoftClipped}}{\emph{N}}{}
Checks SoftClipped read segment for errors. called when cigarstring = S:N

\end{fulllineitems}

\index{next() (proto\_err.errorCount.errorReader method)}

\begin{fulllineitems}
\phantomsection\label{index:proto_err.errorCount.errorReader.next}\pysiglinewithargsret{\bfcode{next}}{}{}
Returns the next error in the samfile aligned read

\end{fulllineitems}

\index{readNext() (proto\_err.errorCount.errorReader method)}

\begin{fulllineitems}
\phantomsection\label{index:proto_err.errorCount.errorReader.readNext}\pysiglinewithargsret{\bfcode{readNext}}{}{}
Iterates to the next read in samfile

\end{fulllineitems}

\index{refRead (proto\_err.errorCount.errorReader attribute)}

\begin{fulllineitems}
\phantomsection\label{index:proto_err.errorCount.errorReader.refRead}\pysigline{\bfcode{refRead}\strong{ None}}
Get the reference sequence the read is aligned to

\end{fulllineitems}


\end{fulllineitems}



\chapter{Indices and tables}
\label{index:indices-and-tables}\begin{itemize}
\item {} 
\emph{genindex}

\item {} 
\emph{modindex}

\item {} 
\emph{search}

\end{itemize}


\renewcommand{\indexname}{Python Module Index}
\begin{theindex}
\def\bigletter#1{{\Large\sffamily#1}\nopagebreak\vspace{1mm}}
\bigletter{p}
\item {\texttt{proto\_err}}, \pageref{index:module-proto_err}
\item {\texttt{proto\_err.errorCount}}, \pageref{index:module-proto_err.errorCount}
\item {\texttt{proto\_err.simulation}}, \pageref{index:module-proto_err.simulation}
\end{theindex}

\renewcommand{\indexname}{Index}
\printindex
\end{document}
