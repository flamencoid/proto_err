%% Title
%% Template for proto_err reports
%% Phelim Bradley - psb31
%% 

%% Text following a percent sign (%) until the end of line is treated
%% as a comment.

\documentclass[a4paper,12pt]{article}

%%%%%%%%%%%%%%%%%%%%%%%%%%%%%%%%%%%%%%%%%%%%%%%%%%%%%%%%%%%%%%%%%%%%%%
%% This section is called the preamble, where we can specify which
%% latex packages we required.  Most (but not of all) of the packages
%% below should be fairly standard in most latex documents.  The
%% exception is xspace and the new \latex command, which you probably
%% do not need.
%%%%%%%%%%%%%%%%%%%%%%%%%%%%%%%%%%%%%%%%%%%%%%%%%%%%%%%%%%%%%%%%%%%%%%

%% Bibliography style:
\usepackage{mathtools}
\usepackage{mathptmx}           % Use the Times font.
\usepackage{graphicx}           % Needed for including graphics.
\usepackage{url}                % Facility for activating URLs.
\usepackage{xspace,color}
\usepackage{enumerate}
%\usepackage{csvsimple}
\usepackage{caption}
\usepackage{amsmath}
\usepackage{amsfonts}
\usepackage{amssymb}
\usepackage{subfig}
\usepackage{tabularx}
\usepackage{listings}
 \captionsetup{justification=centering,singlelinecheck=on}




%% Set the paper size to be A4, with a 2cm margin 
%% all around the page.
\usepackage[a4paper,margin=2cm]{geometry}

%% Natbib is a popular style for formatting references.
\usepackage{natbib}
%% bibpunct sets the punctuation used for formatting citations.
%\bibpunct{[}{]}{;}{a}{,}{,}

%% This is an example of a new macro that I've created to save me
%% having to type \LaTeX each time.  The xspace command provides space
%% after the word LaTeX where appropriate.
\usepackage{xspace}
\providecommand*{\latex}{\LaTeX\xspace}

%Make some shortcuts

\def\be{\begin{equation}}
\def\ee{\end{equation}}

\def\bea{\begin{eqnarray}}
\def\eea{\end{eqnarray}}


\def\toprule{\hline}
\def\midrule{\hline}
\def\bottomrule{\hline}
%%%%%%%%%%%%%%%%%%%%%%%%%%%%%%%%%%%%%%%%%%%%%%%%%%%%%%%%%%%%%%%%%%%%%%
%% Start of the document.
%%%%%%%%%%%%%%%%%%%%%%%%%%%%%%%%%%%%%%%%%%%%%%%%%%%%%%%%%%%%%%%%%%%%%%

\begin{document}




\author{Proto\_err}
\date{\today}
\title{  Run ID : {{runID}}   }
\maketitle

\section{MetaData}

%\begin{verbatim}
	{{options}}
%\end{verbatim}
\begin{verbatim}
{{refRead}}
{{alignedRead}}
\end{verbatim}


\section{Quality Score Distributions}


	\begin{figure}[!htbp]
	  \centering
	  \includegraphics[width=10cm]{{{SNPQualCalibration}}}
	  \caption{Reported vs Empirical Quality Scores, SNPs}
	  \label{fig:SNPQual}
	\end{figure}
	\begin{figure}[!htbp]
	  \centering
	  \includegraphics[width=10cm]{{{INSQualCalibration}}}
	  \caption{Reported vs Empirical Quality Scores, Insertions}
	  \label{fig:INSQual}
	\end{figure}
	\begin{figure}[!htbp]
	  \centering
	  \includegraphics[width=10cm]{{{DELQualCalibration}}}
	  \caption{Reported vs Empirical Quality Scores, Deletions}
	  \label{fig:DELQual}
	\end{figure}	

\begin{figure}[!htbp]
	  \centering
	  \includegraphics[width=10cm]{{{QualDistribution_hist}}}
	  \caption{Distribution of Quality Scores along}
	  \label{fig:QualDistribution_hist}
\end{figure}
	
\section{Error Bias Statistics}

	\begin{figure}[!htbp]
	  \centering
	  \includegraphics[width=10cm]{{{errorDistribution_hist}}}
	  \caption{Distribution of errors along}
	  \label{fig:errorDistribution_histl}
	\end{figure}	
\begin{figure}[!htbp]
	  \centering
	  \includegraphics[width=10cm]{{{SNP_observed_simulated_expected_transition}}}
	  \caption{SNP Transitions}
	  \label{fig:SNP_observed_simulated_expected_transition}
\end{figure}
	

	
\section{INDELs}

	\begin{figure}[!htbp]
	  \centering
	  \includegraphics[width=10cm]{{{deletion_size_bar}}}
	  \caption{Distribution of Deletion Sizes}
	  \label{fig:deletion_size_bar}
	\end{figure}	
\begin{figure}[!htbp]
	  \centering
	  \includegraphics[width=10cm]{{{insertion_size_bar}}}
	  \caption{Distribution of Insertion Sizes}
	  \label{fig:insertion_size_bar}
\end{figure}
	

\section{Summary}
\begin{table}[htbp]
  \centering
{{errorsDF}} 
\caption{Count of errors simulated and found in the resulting aligned samfile}
%  \label{tab:StudentResults}
\end{table}

\begin{table}[htbp]
  \centering
{{readCount1}}  
\caption{Alignment read statistics}
%  \label{tab:StudentResults}
\end{table}

\begin{table}[htbp]
  \centering
{{readCount2}} 
\caption{Alignment base statistics}
%  \label{tab:StudentResults}
\end{table}

\begin{table}[htbp]
  \centering
{{significantContext1}} 
\caption{Context with the most significant difference between expected and observed count}
%  \label{tab:StudentResults}
\end{table}

\begin{table}[htbp]
  \centering
{{significantContext2}} 
\caption{Context with the most significant difference between expected and observed count}
%  \label{tab:StudentResults}
\end{table}

\begin{table}[htbp]
  \centering
{{significantQualContext1}} 
\caption{Quality Score Context with the most significant difference between expected and observed count}
%  \label{tab:StudentResults}
\end{table}


\begin{table}[htbp]
  \centering
{{significantQualContext2}} 
\caption{Quality Score Context with the most significant difference between expected and observed count}
%  \label{tab:StudentResults}
\end{table}


\begin{table}[htbp]
  \centering
{{lowQualScore1}} 
\caption{Contexts with the lowest mean quality score}
%  \label{tab:StudentResults}
\end{table}


\begin{table}[htbp]
  \centering
{{lowQualScore2}} 
\caption{Contexts with the lowest mean quality score}
%  \label{tab:StudentResults}
\end{table}

\begin{table}[htbp]
  \centering
{{inssignificantContext1}} 
\caption{Context before Insertions with the most significant difference between expected and observed count}
%  \label{tab:StudentResults}
\end{table}



\begin{table}[htbp]
  \centering
{{delsignificantContext1}} 
\caption{Context before Deletions with the most significant difference between expected and observed count}
%  \label{tab:StudentResults}
\end{table}





%
%\begin{figure}[!ht]
%	\def\tabularxcolumn#1{m{#1}}
%\begin{tabularx}{\linewidth}{@{}cXX@{}}
%\centering
%
%\begin{tabular}{cc}
%\subfloat[Evolution of Secondary Structure]{\includegraphics[width=10cm]{apo_md_timeline_full.pdf}} 
%   & \subfloat[Secondary structure change. Transparent orange shows original structure]{\includegraphics[width=7cm]{PartAQ6StructureChange4.png}}\\
%\end{tabular}
%
%\end{tabularx}
%
%\caption{Secondary structure change}
%\label{fig:TimeLine}
%\end{figure}


%% Table generated by Excel2LaTeX from sheet 'resultsFinal.csv'
%\begin{table}[htbp]
%  \centering
%    \begin{tabular}{|r|r|r|r|}
%    \hline
%    student & score & grade & rank \\
%   \hline
%    1     & 19    & B     & 7 \\
%    2     & 24    & A     & 2 \\
%    3     & 14    & D     & 10 \\
%    4     & 17    & C     & 9 \\
%    5     & 20    & B     & 5 \\
%    6     & 23    & A     & 3 \\
%    7     & 29    & A     & 1 \\
%    8     & 7     & F     & 12 \\
%    9     & 22    & A     & 4 \\
%    10    & 20    & B     & 5 \\
%    11    & 9     & F     & 11 \\
%    12    & 18    & B     & 8 \\
%    \hline
%    \end{tabular}
%  \caption{Tabulated results of 12 students}
%  \label{tab:StudentResults}
%\end{table}
%	
%	\begin{figure}[htbp]
%	  \centering
%	  \includegraphics[width=10cm]{HistogramofComparisionScore.pdf}
%	  \caption{Histogram of Comparison Scores}
%	  \label{fig:histcompscore}
%	\end{figure}



%%%%%%%%%%%%%%%%%%%%%%%%%%%%%%%%%%%%%%%%%%%%%%%%%%%%%%%%%%%%%%%%%%%%%%
%% Finally we specify the format required for our references and the
%% name of the bibtex file where our references should be taken from.
\bibliographystyle{plain}	% (uses file "plain.bst")
\bibliography{template}	
%%%%%%%%%%%%%%%%%%%%%%%%%%%%%%%%%%%%%%%%%%%%%%%%%%%%%%%%%%%%%%%%%%%%%%
\newpage
\appendix




\end{document}

%%%%%%%%%%%%%%%%%%%%%%%%%%%%%%%%%%%%%%%%%%%%%%%%%%%%%%%%%%%%%%%%%%%%%%
%% The end.
%%%%%%%%%%%%%%%%%%%%%%%%%%%%%%%%%%%%%%%%%%%%%%%%%%%%%%%%%%%%%%%%%%%%%%
